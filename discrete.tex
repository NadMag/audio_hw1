\section{Discrete Signals}
\subsection{Aliasing}
\begin{enumerate}
  \item The set of signals which would be aliased to 10KHz with a sampling frequency of 8KHz is given by $\sizeof{f - 10} = 8k$ for some $k\in N$.
  All such signals would be aliased to $\textbf{2KHz}$

  \item We can prevent the aliasing by passing the analog signal through an anti-aliasing filter of bandwidth 4KHz (Nyquist frequency).
  prior to sampling
\end{enumerate}

\subsection{Stereo Hearing}
When playing both channels in sync we hear the original recording.
When shifting the left track 2 ms ahead the sound appears to travel from the right ear to the left and vise versa when shifting the right track.
This occurs because when a sound source is moving relative to our ears there is a slight phase shift between the signal received in each ear.
In this case we emulate the phase shift by moving one track 2ms ahead.


\subsection{$\mathcal{Z}$ Transform}

\subsubsection{Convolution Property}
\begin{equation}
  \begin{aligned}
    \mathcal{Z}\left( x_1[n] * x_2[n] \right) &= \mathcal{Z}\left( \sum_{m \in \mathbb{Z}} x_1[m] \cdot x_2[n-m] \right) \\
    &= \sum_{n \in \mathbb{Z}} z^{-n} \sum_{m \in \mathbb{Z}} x_1[m] \cdot x_2[n-m] \\
    &= \sum_{m \in \mathbb{Z}} \sum_{n \in \mathbb{Z}} x_1[m] \cdot x_2[n-m] z^{-(n-m)} z^{-m} \\
    &= \sum_{m \in \mathbb{Z}} x_1[m] z^{-m} \sum_{n \in \mathbb{Z}} x_2[n-m] z^{-(n-m)} \\
    &= \sum_{m \in \mathbb{Z}} x_1[m] z^{-m} \sum_{n \in \mathbb{Z}} x_2[n] z^{-n} \\
    &= X_1[z] \cdot X_2[z]
  \end{aligned}
\end{equation}

\subsubsection{Time Domain Scaling}
Assuming our signal is unilateral:
\begin{equation}
  \begin{aligned}
    \mathcal{Z}\left(a^n x[n] \right) &= \sum_{n \in \mathbb{N}} a^n x[n] z^{-n} \\
    &= \sum_{n \in \mathbb{N}} x[n] \left(\frac{z}{a}\right)^{-n} \\
    &= X\left( \frac{z}{a} \right)
  \end{aligned}
\end{equation}


\subsection{DTFS}
\begin{enumerate}
  \item We compute the period of the discrete sinusoid
  \begin{equation}
    \begin{aligned}
      N_0 &= \min_{m \in \mathbb{N} \; s.t. m\left(\frac{2\pi}{0.1\pi}\right) \in \mathbb{N}} m\left(\frac{2\pi}{0.1\pi}\right)  \\
      &= 20
    \end{aligned}
  \end{equation}

  \item The sinusoid can easily be expressed as a sum of harmonic exponentials but for completeness we derive its DFTS step by step. Let us chose the 0-centered interval of length $N_0$ $S = \{-9, -8, ..., 10\}$.
  We compute the DTFS coefficients:
  \begin{equation}
    \begin{aligned}
      \mathcal{D}_r &= \frac{1}{20} \sum_{n \in S} \sin\left( 0.1\pi n \right) e^{-0.1 \pi i r n} \\
      &= \frac{1}{20} \sum_{n \in S} \frac{1}{2i} \left( e^{0.1\pi n i} - e^{-0.1\pi n i} \right) e^{-0.1 \pi i r n} \\
      &= \frac{1}{40i} \sum_{n \in S} \left( e^{0.1\pi n i \left(1-r\right)} - e^{-0.1\pi n i \left(1 + r\right)} \right) \\
      &= \frac{1}{40i} \left(\sum_{n \in S} e^{0.1\pi n i \left(1-r\right)} - \sum_{n \in S} e^{-0.1\pi n i \left(1 + r\right)} \right) \\
    \end{aligned}
  \end{equation}
  Now noticing that each sum is a geometric progression of an harmonic exponential across its period we have:
  \begin{equation}
    \mathcal{D}_r = 
    \begin{cases}
      \frac{1}{2i} &  r=1  \\
      -\frac{1}{2i} &   r=-1 \\
      0 & \text{otherwise}
   \end{cases}
  \end{equation}
  Plugging this into the DTFS we get
  \begin{equation}
    x[n] = \frac{1}{2i} \left( e^{0.1 \pi i n} +  e^{-0.1 \pi i n} \right)
  \end{equation}
\end{enumerate}
